\documentclass[a4paper,oneside,10.5pt]{article}
\usepackage{../../TeXMacros/personal}

\hypersetup{
pdfauthor = {Jorge Bravo},
pdfsubject = {EDO},
pdftitle = {EDOs de Primer Orden},
pdfkeywords = {},
}

\setlength\parindent{0px}
\begin{document}


\begin{center}
{\Large \textsc{EDOs de Primer Orden}}\\
\vspace{1em}
\textsc{Ayudante: Jorge Bravo}\\
\end{center}

\section*{EDO Lineal de Primer Orden}
Una ecuación diferencial de grado $1$ se dirá lineal si es de la siguiente forma para funciones $f,g \in \mathcal{C}$

\begin{equation}
  \label{edo lineal}
  y' + f(x)y = g(x)
\end{equation}

Dada una EDO lineal de primer orden se define el factor integrador como
\begin{equation*}
  \mu(x) = e^{\int f(x) dx}
\end{equation*}

Luego la solución al problema $(\ref{edo lineal})$ viene dada por la formula de Leibniz, la cual es
\begin{equation*}
  y(x) = \frac{1}{\mu(x)} (\int \mu(x) g(x) dx)
\end{equation*}

Para resolver este tipo de EDO's entonces lo unico que hay que hacer es reconocer que es una EDO lineal de primer orden y despues aplicar la formula de Leibniz.

\begin{ejemplo}
  Resuelva la siguiente EDO
  \begin{equation*}
    ty' + 2y = t^{2} - t  + 1
  \end{equation*}

  Lo primero que haremos sera dejarla en la forma canonica, es decir con el $y'$ solo, luego dividimos por $t$ y obtenemos que
  \begin{equation*}
    y' + \frac{2}{t}y = t - 1 + \frac{1}{t}
  \end{equation*}

  Luego la edo ya esta en su forma lineal para aplicar Leibniz. Calculemos el factor integrador
  \begin{equation*}
    \mu(t) = e^{\int \frac{2}{t} dt} = e^{2 \ln(|t|)} = t^{2}
  \end{equation*}

  Entonces la solucion a la EDO viene dada por
  \begin{equation*}
    y(t) = \frac{1}{\mu(t)}(\int \mu(t)(t - 1 + \frac{1}{t}) dt) = t^{-2}(\int t^{3} - t^{2} + t dt) = t^{-2}(\frac{1}{4} t^{4} - \frac{1}{3}t^{3} + \frac{1}{2}t^{2} + C)
  \end{equation*}

  Por lo tanto la solucion general viene dada por
  \begin{equation*}
    y(t) = \frac{1}{4}t^{2} - \frac{1}{3}t + \frac{1}{2} + Ct^{-2}
  \end{equation*}


\end{ejemplo}

\section*{Ecuaci\'on de Bernoulli}
La EDO de Bernoulli es una ecuacion diferencial de primer orden con la siguiente estructura
\begin{equation*}
  y' + f(x) y = g(x)y^{\alpha}, \alpha \neq 1
\end{equation*}
Con $f, g \in \mathcal{C}$

Para resolver esta EDO seguiremos los siguientes pasos
\begin{enumerate}
  \item Reconocer que es una EDO de Bernoulli
  \item Multiplicar por $y^{-\alpha}$ la ecuacion, para dejarla de la siguiente forma
        \begin{equation*}
          \label{cambio}
          y^{-\alpha}y' + f(x)y^{1 - \alpha} = g(x)
        \end{equation*}
  \item Hacer el cambio de variable $z = y^{1 - \alpha}$, derivando obtenemos que
        \begin{equation*}
          z' = (1 - \alpha)y^{-\alpha}y' \implies \frac{z'}{(1 - \alpha)y^{-\alpha}} = y'
        \end{equation*}
  \item Reemplazamos en $(\ref{cambio})$ con la nueva variable
        \begin{equation*}
          \frac{1}{1 - \alpha} z' + f(x)z = g(x)
        \end{equation*}
  \item Dejamos $z'$ ``despejado''
        \begin{equation*}
          z' + (1 - \alpha) f(x) z = g(x)(1 - \alpha)
        \end{equation*}
  \item Resolver la EDO como una EDO lineal de primer orden
  \item Deshacer el cambio de variable.
\end{enumerate}

\begin{ejemplo}
  Resuelva la siguiente Ecuacion diferencial
  \begin{equation*}
    y' + \frac{4}{x}y = x^{3}y^{2}, x > 0
  \end{equation*}
\end{ejemplo}

\begin{enumerate}
  \item Lo primero que haremos sera reconocer que es una EDO de Bernoulli con $\alpha = 2$, $f(x) = \frac{4}{x}$ y $g(x) = x^{3}$.
  \item Multiplicamos por $y^{-2}$
        \begin{equation*}
          y^{-2}y' + \frac{4}{x}y^{-1} = x^{3}
        \end{equation*}
  \item Hacemos el cambio de variable $z = y^{1 - 2} = y^{-1}$, luego tenemos que $z' = -y^{-2}y' \implies y' = -y^{2}z'$
  \item Reemplazamos en la EDO
        \begin{equation*}
          -y^{-2}y^{2}z' + \frac{4}{x} z = x^{3} \iff -z' + \frac{4}{x} z = x^{3}
        \end{equation*}

  \item Despejamos $z'$, en este caso solo hay que multiplicar por $-1$
        \begin{equation*}
          z' - \frac{4}{x}z = -x^{3}
        \end{equation*}

  \item Ahora resolvemos la EDO lineal de primer orden que nos queda. Notemos que el factor integrante viene dado por
        \begin{equation*}
          \mu(x) = e^{\int -\frac{4}{x} dx}
        \end{equation*}

        Calculemos la integral
        \begin{equation*}
          \int - \frac{4}{x} dx = - 4 \int \frac{1}{x} dx = - 4 \ln(x) = \ln(x^{-4})
        \end{equation*}

        Luego el factor integrate es
        \begin{equation*}
          \mu(x) = e^{\ln(x^{-4})} = x^{-4}
        \end{equation*}

        Por la formula de Leibniz, tenemos que la solucion a la EDO viene dada por
        \begin{equation*}
          z(x) = \frac{1}{\mu(x)} (\int \mu(x) g(x) dx) = x^{4}(\int x^{-4}(-x^{3}))
        \end{equation*}

        Hacemos la integral
        \begin{equation*}
          \int x^{-4}(-x^{3}) dx = - \int x^{-1} dx = - \int \frac{1}{x} dx = - \ln (x) + C
        \end{equation*}

        Por lo que la solucion a la edo en terminos de $z$ viene dada por
        \begin{equation*}
          z(x) = x^{4}(C - \ln(x))
        \end{equation*}

  \item Deshacemos el cambio de variable
        \begin{equation*}
          y^{-1}(x) = z(x) = x^{4}(C - \ln(x)) \implies y(x) = \frac{1}{x^{4}(C - \ln(x))}
        \end{equation*}

\end{enumerate}

\section*{EDO de Ricatti}
La edo de Ricatti es una Ecuacion diferencial de primer orden \textbf{no} lineal que tiene la siguiente forma
\begin{equation*}
  y' + f(x) y + g(x) y^{2} = h(x)
\end{equation*}
con $f, g, h \in \mathcal{C}$

Para resolver este tipo de EDO's, lo primero que necesitamos es conocer \textbf{una} solucion a la EDO, es decir necesitamos una funcion $u(x)$ tal que cuando la reemplazamos en la ecuacion esta se satisfaga. Una vez tenemos esto podemos encontrar el resto de las soluciones con los siguientes pasos

\begin{enumerate}
  \item Definimos el siguiente cambio de variable
        \begin{equation*}
          y = u + \frac{1}{v}
        \end{equation*}

        Donde $v$ es la nueva variable y $u$ es la solucion que conocemos
  \item Derivamos para despejar $y'$
        \begin{equation*}
          y' = u'  - \frac{v'}{v^{2}}
        \end{equation*}

  \item Reemplazamos en la EDO
        \begin{equation*}
          (u' - \frac{v'}{v^{2}}) + f(x)(u + \frac{1}{v}) + g(x)(u + \frac{1}{v})^{2} = h(x)
        \end{equation*}

  \item Reordenamos la edo
        \begin{align*}
          u' - \frac{v'}{v^{2}} + f(x)u + \frac{f(x)}{v} + g(x)(u^{2} + \frac{2u}{v} + \frac{1}{v^{2}}) &= h(x)\\
          \iff u' - \frac{v'}{v^{2}} + f(x)u + \frac{f(x)}{v} + g(x)u^{2} + g(x) \frac{2u}{v} + \frac{g(x)}{v^{2}} &= h(x)\\
          \iff (u' + f(x)u + g(x)u^{2}) - \frac{v'}{v^{2}} + \frac{f(x)}{v} + g(x) \frac{2u}{v} + \frac{g(x)}{v^{2}} &= h(x)
        \end{align*}

        Ahora recordamos que $u$ satisface la edo y por tanto el parentesis de la izquierda es igual a $h(x)$

        \begin{align*}
          h(x) - \frac{v'}{v^{2}} + \frac{f(x)}{v} + g(x) \frac{2u}{v} + \frac{g(x)}{v^{2}} &= h(x)\\
          \iff - \frac{v'}{v^{2}} + \frac{f(x)}{v} + g(x) \frac{2u}{v} + \frac{g(x)}{v^{2}} &= 0\\
          \iff v' - f(x)v - 2u g(x)v- g(x) &= 0\\
          \iff v' + (-f(x) - 2ug(x))v &= g(x)
        \end{align*}
  \item Resolvemos la EDO lineal que nos queda
  \item Devolvemos el cambio de variable
\end{enumerate}

\begin{obs}
 Hay profesores que dejan saltar desde $1.$ hasta el final de $4.$ directamente, por lo que no es necesario hacer todo el trabajo, solo aprenderse la forma de la EDO despues de hacer el cambio de variable.
\end{obs}

\begin{ejemplo}

\end{ejemplo}

\section*{Problemas}
\begin{prob}
  Resuelva la siguiente EDO
  \begin{equation*}
    y' + ty = 5t
  \end{equation*}
\end{prob}
\begin{sol}
  Esta es una EDO lineal que ya esta en su forma canonica. Esta tiene $f(t) = t$ y $g(t) = 5t$, luego su factor integrante viene dado por
  \begin{equation*}
    \mu(t) = e^{\int f(t) dt} = e^{\int t dt} = e^{\frac{t^{2}}{2}}
  \end{equation*}

  Luego la solucion general viene dada por
  \begin{equation*}
    y(t) = \frac{1}{\mu(t)}(\int \mu(t) g(t) dt) = e^{-\frac{t^{2}}{2}} (\int e^{\frac{t^{2}}{2}} 5t dt)
  \end{equation*}

  Calculemos la integral, hacemos el cambio de variable $u = \frac{t^{2}}{2}$, luego $du = t dt$.
  Entonces la integral nos queda
  \begin{equation*}
    \int e^{\frac{t^{2}}{2}} 5t dt = 5 \int e^{u} du = 5e^{u} + C= 5e^{\frac{t^{2}}{2}} + C
  \end{equation*}

  Por lo tanto la solucion es
  \begin{equation*}
    y(t) = e^{-\frac{t^{2}}{2}}(5e^{\frac{t^{2}}{2}} + C) = 5 + Ce^{-\frac{t^{2}}{2}}
  \end{equation*}
\end{sol}

\begin{prob}[control]
  Resolver la ecuacion con valor inicial (Bernoulli):
  \begin{equation*}
    y' = \frac{1}{x}y - \frac{2x}{y}
  \end{equation*}

  Con la condicion inicial
  \begin{equation*}
    y(1) = 1
  \end{equation*}

\end{prob}
\begin{sol}
  Ya que sabemos que es una ecuacion diferencial de Bernoulli, primero la pondremos en su forma ``canonica''
  \begin{equation*}
    y' - \frac{1}{x}y = -2x y^{-1}
  \end{equation*}

  Luego esta es una ecuacion de Bernoulli con $\alpha = -1$, por lo que multiplicaremos por $y$
  \begin{equation*}
    yy' - \frac{1}{x}y^{2} = -2x
  \end{equation*}

  Hacemos el cambio de variable $z = y^{2}$, luego $z' = 2yy' \iff y' = \frac{z'}{2y}$
  \begin{equation*}
    y \frac{z'}{2y} - \frac{1}{x} z = -2x \iff \frac{1}{2}z' - \frac{1}{x}z = -2x
  \end{equation*}

  Despejamos el $z'$, luego
  \begin{equation*}
    z' - \frac{2}{x}z = -4x
  \end{equation*}

  Esta es una EDO lineal de primer orden, la resolvemos con la formula de Leibniz, el factor integrante viene dado por
  \begin{equation*}
    \mu(x) = e^{\int -\frac{2}{x} dx}
  \end{equation*}

  Hacemos la integral, luego
  \begin{equation*}
    \int -\frac{2}{x} dx = -2 \int \frac{1}{x} dx = -2 \ln (x) = \ln(x^{-2})
  \end{equation*}

  Por lo tanto el factor integrante es
  \begin{equation*}
    \mu(x) = e^{\ln(x^{-2})} = x^{-2}
  \end{equation*}

  Luego por la formula de Leibniz, la solucion a la EDO viene dada por
  \begin{equation*}
    z(x) = \frac{1}{\mu(x)}(\int \mu(x)(-4x) dx) = -x^{2}\int x^{-1} dx = -4x^{2} (\ln(x) + C)
  \end{equation*}

  Ahora deshacemos el cambio de variable, recordamos que
  \begin{equation*}
    y^{2} = z = -x^{2} \ln(x) + C \iff y = \sqrt{-4x^{2} \ln(x) + Cx^{2}}
  \end{equation*}

  Ahora evaluamos en $1$ para encontrar la la constante
  \begin{equation*}
    1 = y(1) = \sqrt{C} \iff C = 1
  \end{equation*}

  Luego la solucion general al problema de valor inicial viene dado por
  \begin{equation*}
    y = \sqrt{-4x^{2}\ln(x) + x^{2}}
  \end{equation*}


\end{sol}

\end{document}
