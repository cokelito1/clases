\documentclass[a4paper,oneside,10.5pt]{article}
\usepackage{../../TeXMacros/personal}

\hypersetup{
pdfauthor = {Jorge Bravo},
pdfsubject = {EDO},
pdftitle = {EDOs de Primer Orden},
pdfkeywords = {},
}

\setlength\parindent{0px}
\begin{document}


\begin{center}
{\Large \textsc{EDOs de Primer Orden}}\\
\vspace{1em}
\textsc{Ayudante: Jorge Bravo}\\
\end{center}

\section*{EDO Lineal de Primer Orden}
Una ecuación diferencial de grado $1$ se dirá lineal si es de la siguiente forma para funciones $f,g \in \mathcal{C}$

\begin{equation}
  \label{edo lineal}
  y' + f(x)y = g(x)
\end{equation}

Dada una EDO lineal de primer orden se define el factor integrador como
\begin{equation*}
  \mu(x) = e^{\int f(x) dx}
\end{equation*}

Luego la solución al problema $(\ref{edo lineal})$ viene dada por la formula de Leibniz, la cual es
\begin{equation*}
  y(x) = \frac{1}{\mu(x)} (\int \mu(x) g(x) dx)
\end{equation*}

Para resolver este tipo de EDO's entonces lo unico que hay que hacer es reconocer que es una EDO lineal de primer orden y despues aplicar la formula de Leibniz.

\begin{ejemplo}

\end{ejemplo}

<<<<<<< HEAD
\section*{Ecuaci\'on de Bernoulli}
La EDO de Bernoulli es una ecuacion diferencial de primer orden con la siguiente estructura
\begin{equation*}
  y' + f(x) y = g(x)y^{\alpha}, \alpha \neq 1
\end{equation*}

Para resolver esta EDO seguiremos los siguientes pasos
\begin{enumerate}
  \item Reconocer que es una EDO de Bernoulli
  \item Multiplicar por $y^{-\alpha}$ la ecuacion, para dejarla de la siguiente forma
        \begin{equation*}
          \label{cambio}
          y^{-\alpha}y' + f(x)y^{1 - \alpha} = g(x)
        \end{equation*}
  \item Hacer el cambio de variable $z = y^{1 - \alpha}$, derivando obtenemos que
        \begin{equation*}
          z' = (1 - \alpha)y^{-\alpha}y' \implies \frac{z'}{(1 - \alpha)y^{-\alpha}} = y'
        \end{equation*}
  \item Reemplazamos en $(\ref{cambio})$ con la nueva variable
        \begin{equation*}
          \frac{1}{1 - \alpha} z' + f(x)z = g(x)
        \end{equation*}
  \item Dejamos $z'$ ``despejado''
        \begin{equation*}
          z' + (1 - \alpha) f(x) z = g(x)(1 - \alpha)
        \end{equation*}
  \item Resolver la EDO como una EDO lineal de primer orden
  \item Deshacer el cambio de variable.
\end{enumerate}

\begin{ejemplo}
  Resuelva la siguiente Ecuacion diferencial
  \begin{equation*}
    y' + \frac{4}{x}y = x^{3}y^{2}, x > 0
  \end{equation*}
\end{ejemplo}

\begin{enumerate}
  \item Lo primero que haremos sera reconocer que es una EDO de Bernoulli con $\alpha = 2$, $f(x) = \frac{4}{x}$ y $g(x) = x^{3}$.
  \item Multiplicamos por $y^{-2}$
        \begin{equation*}
          y^{-2}y' + \frac{4}{x}y^{-1} = x^{3}
        \end{equation*}
  \item Hacemos el cambio de variable $z = y^{1 - 2} = y^{-1}$, luego tenemos que $z' = -y^{-2}y' \implies y' = -y^{2}z'$
  \item Reemplazamos en la EDO
        \begin{equation*}
          -y^{-2}y^{2}z' + \frac{4}{x} z = x^{3} \iff -z' + \frac{4}{x} z = x^{3}
        \end{equation*}

  \item Despejamos $z'$, en este caso solo hay que multiplicar por $-1$
        \begin{equation*}
          z' - \frac{4}{x}z = -x^{3}
        \end{equation*}

  \item Ahora resolvemos la EDO lineal de primer orden que nos queda. Notemos que el factor integrante viene dado por
        \begin{equation*}
          \mu(x) = e^{\int -\frac{4}{x} dx}
        \end{equation*}

        Calculemos la integral
        \begin{equation*}
          \int - \frac{4}{x} dx = - 4 \int \frac{1}{x} dx = - 4 \ln(x) = \ln(x^{-4})
        \end{equation*}

        Luego el factor integrate es
        \begin{equation*}
          \mu(x) = e^{\ln(x^{-4})} = x^{-4}
        \end{equation*}

        Por la formula de Leibniz, tenemos que la solucion a la EDO viene dada por
        \begin{equation*}
          z(x) = \frac{1}{\mu(x)} (\int \mu(x) g(x) dx) = x^{4}(\int x^{-4}(-x^{3}))
        \end{equation*}

        Hacemos la integral
        \begin{equation*}
          \int x^{-4}(-x^{3}) dx = - \int x^{-1} dx = - \int \frac{1}{x} dx = - \ln (x) + C
        \end{equation*}

        Por lo que la solucion a la edo en terminos de $z$ viene dada por
        \begin{equation*}
          z(x) = x^{4}(C - \ln(x))
        \end{equation*}

  \item Deshacemos el cambio de variable
        \begin{equation*}
          y^{-1}(x) = z(x) = x^{4}(C - \ln(x)) \implies y(x) = \frac{1}{x^{4}(C - \ln(x))}
        \end{equation*}




\end{enumerate}

=======
>>>>>>> 49e710b (Se creo la EDO de primer orden)
\end{document}
