\documentclass[a4paper,oneside,10.5pt]{article}
\usepackage{../../TeXMacros/personal}

\hypersetup{
pdfauthor = {Jorge Bravo},
pdfsubject = {EDO},
pdftitle = {EDOs de Primer Orden},
pdfkeywords = {},
}

\setlength\parindent{0px}
\begin{document}


\begin{center}
{\Large \textsc{EDOs de Primer Orden}}\\
\vspace{1em}
\textsc{Ayudante: Jorge Bravo}\\
\end{center}

\section*{EDO Lineal de Primer Orden}
Una ecuación diferencial de grado $1$ se dirá lineal si es de la siguiente forma para funciones $f,g \in \mathcal{C}$

\begin{equation}
  \label{edo lineal}
  y' + f(x)y = g(x)
\end{equation}

Dada una EDO lineal de primer orden se define el factor integrador como
\begin{equation*}
  \mu(x) = e^{\int f(x) dx}
\end{equation*}

Luego la solución al problema $(\ref{edo lineal})$ viene dada por la formula de Leibniz, la cual es
\begin{equation*}
  y(x) = \frac{1}{\mu(x)} (\int \mu(x) g(x) dx)
\end{equation*}

Para resolver este tipo de EDO's entonces lo unico que hay que hacer es reconocer que es una EDO lineal de primer orden y despues aplicar la formula de Leibniz.

\begin{ejemplo}

\end{ejemplo}

\end{document}
