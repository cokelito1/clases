\documentclass[a4paper,oneside,10.5pt]{article}
\usepackage{../../TeXMacros/personal}

\hypersetup{
pdfauthor = {Jorge Bravo},
pdfsubject = {EDO},
pdftitle = {Introduccion a las EDO},
pdfkeywords = {},
}

\setlength\parindent{0px}
\begin{document}


\begin{center}
{\Large \textsc{Introduccion EDO}}\\
\vspace{1em}
\textsc{Ayudante: Jorge Bravo}\\
\end{center}

\section*{Preliminares y definiciones basicas}
Eventualmente escribire esto

\section*{EDO separable}
Una EDO separable es una ecuación diferencial ordinaria de grado $1$ que se puede escribir de la siguiente forma
\begin{equation*}
y' = f(x)g(y)
\end{equation*}

con $f, g$ funciones continuas. Notemos que la función $f$ \textbf{no} puede depender de $y$, análogamente $g$ no puede depender de $x$. Es decir nos queda algo que solo depende de $x$ multiplicado por algo que solo depende de $y$.

\begin{ejemplo}
  La ecuación diferencial dada por
  \begin{equation*}
    y' = 6y^{2}x
  \end{equation*}
  Es separable, pues la podemos escribir como

  \begin{equation*}
    y' = (6y^{2})(x)
  \end{equation*}
\end{ejemplo}

\begin{ejemplo}
La EDO dada por
\begin{equation*}
y' = \frac{xy^{3}}{\sqrt{1+x^{2}}}
\end{equation*}

Es separable pues la podemos escribir de la forma
\begin{equation*}
y' = (\frac{x}{\sqrt{1+x^{2}}})(y^{3})
\end{equation*}
\end{ejemplo}

\begin{ejemplo}
  La EDO dada por
  \begin{equation*}
    y' = e^{x - y}
  \end{equation*}

  Es separable pues la podemos escribir como
  \begin{equation*}
    y' = e^{x - y} = e^{x}e^{-y}
  \end{equation*}
\end{ejemplo}

Para resolver una EDO separable haremos un truco, el cual se puede justificar de manera formal pero no va al caso ver porque. El truco es el siguiente.

\begin{enumerate}
  \item Escribir la ecuación diferencial en su forma separada, es decir dejarla de la forma
        \begin{equation*}
          y' = f(x)g(y)
        \end{equation*}
  \item Cambiar la notación $y'$ por $\frac{dy}{dx}$, es decir escribir la EDO de la siguiente forma
        \begin{equation*}
          \frac{dy}{dx} = f(x) g(y)
        \end{equation*}

  \item Dejar todo lo que tenga $y$ a un lado y todo lo que tenga $x$ al otro, para esto es necesario multiplicar por $dx$, lo hacemos de manera totalmente formal
        \begin{equation*}
          \frac{1}{g(y)} dy = f(x) dx
        \end{equation*}

  \item Integrar ambos lados con respecto a su diferencial
        \begin{equation*}
          \int \frac{1}{g(y)} dy = \int f(x) dx
        \end{equation*}
  \item Una vez se integro ambos lados, despejar $y$.

  \begin{ejemplo}
    Resolvamos la siguiente EDO
    \begin{equation*}
      y' = 6y^{2}x
    \end{equation*}

    Esta ya esta escrita en su forma separada, por lo que cambiamos la notación a la de Leibniz
    \begin{equation*}
      \frac{dy}{dx} = 6y^{2}x
    \end{equation*}

    Dejamos todo lo que tenga $y$ a un lado y todo lo que tenga $x$ al otro
    \begin{equation*}
      \frac{1}{6y^{2}} dy = x dx
    \end{equation*}

    Integramos con respecto a cada variable

    \begin{equation*}
    -\frac{1}{6y} = \int \frac{1}{6y^{2}}  dy = \int x dx = \frac{x^{2}}{2} + C
    \end{equation*}

    Luego obtenemos
    \begin{equation*}
     -\frac{1}{6y} = \frac{x^{2}}{2} + C, C \in \RR
    \end{equation*}

    Por ultimo despejamos $y$
    \begin{equation*}
      -\frac{1}{6y} = \frac{x^{2}}{2} + C \iff -\frac{1}{6} = (\frac{x^{2}}{2} + C)y \iff y = -\frac{1}{6(\frac{x^{2}}{2} + C)}
    \end{equation*}

    Reordenando al final obtenemos que
    \begin{equation*}
      y = -\frac{1}{6x^{2} + 12C}
    \end{equation*}

    Pero como $C$ es arbitraria, $12C$ también lo es y podemos dejarlo como $C$ nada mas.
    Es decir la solución general es
    \begin{equation*}
      y = -\frac{2}{6x^{2} + C}
    \end{equation*}
  \end{ejemplo}

\end{enumerate}
\section*{Problemas}
\begin{prob}
  Resuelva la siguiente EDO
  \begin{equation*}
    y' = -k\sqrt{y}
  \end{equation*}
\end{prob}
\begin{sol}
  La solucion general a la edo viene dada por
  \begin{equation*}
    y(x) = (C - kx)^{2}
  \end{equation*}
\end{sol}

\end{document}
