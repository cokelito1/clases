\documentclass[a4paper,oneside,10.5pt]{article}
\usepackage{../../TeXMacros/personal}

\hypersetup{
pdfauthor = {Jorge Bravo},
pdfsubject = {MAT023 2024},
pdftitle = {Multiplicadores de Lagrange},
pdfkeywords = {},
}

\setlength\parindent{0px}
\begin{document}


\begin{center}
{\Large \textsc{Multiplicadores de Lagrange}}\\
\vspace{1em}
\textsc{Ayudante: Jorge Bravo}\\
\end{center}

\section*{Problemas}
\begin{prob}
    La temperatura $(x, y)$ de una placa de metal es $T(x, y) = 4x^2 - 4xy + y^2$. Una hormiga camina sobre la placa alrededor de una circunferencia centrada en el origen y de radio 5. Cual es la mayor y menor temperatura con la que se encuentra la hormiga?
\end{prob}

\begin{sol}
    Dado que la hormiga se encuentra sobre una circunferencia de radio 5 centrada en el origen, los puntos por los que se mueve deben satisfacer
    \begin{equation*}
        x^2 + y^2 = 5^2
    \end{equation*}

    Además notemos que $T$ es una función de clase $\mathcal{C}^\infty$, por lo tanto podemos ocupar el método de multiplicadores de Lagrange. Defininamos
    \begin{gather*}
        g : \RR^2 \to \RR\\
        (x, y) \mapsto x^2 + y^2 - 5^2
    \end{gather*}

    Ahora queremos resolver el sistema
    \begin{equation*}
        Df(x, y) = \lambda Dg(x, y)
    \end{equation*}

    Calculemos las derivadas.
    \begin{gather*}
        Df(x, y) = \begin{bmatrix} \partial_x f & \partial_y f \end{bmatrix} = \begin{bmatrix}
            8x - 4y  & -4x + 2y
        \end{bmatrix}\\
        Dg(x, y) = \begin{bmatrix}
            \partial_x g & \partial_y g
        \end{bmatrix} = \begin{bmatrix}
            2x & 2y
        \end{bmatrix}
    \end{gather*}

    Luego el nos queda que
    \begin{equation*}
        \begin{bmatrix}
            8x - 4y  & -4x + 2y
        \end{bmatrix} = \begin{bmatrix}
            2\lambda x & 2 \lambda y
        \end{bmatrix}
    \end{equation*}

    Es decir queremos resolver el sistema
    \begin{align*}
        8x - 4y &= 2\lambda x\\
        -4x + 2y &= 2\lambda y\\
        x^2 + y^2 &= 25
    \end{align*}

    Dividiendo la primera ecuacion por $2$ obtenemos
    \begin{align*}
        4x - 2y &= \lambda x\\
        -4x + 2y &= 2\lambda y\\
        x^2 + y^2 &= 25
    \end{align*}

    Sumando la primera ecuacion con la segunda obtenemos
    \begin{equation*}
        0 = \lambda x + 2\lambda y \implies \lambda(x + 2y) = 0
    \end{equation*}

    \textbf{Caso 1: }$\lambda = 0$, si $\lambda = 0$, entonces de la primera ecuacion obtenemos que $2x = y$, reemplazando en la tercera ecuacion obtenemos
    \begin{equation*}
        x^2 + 4x^2 = 25 \implies x = \pm \sqrt{5}
    \end{equation*}

    Por lo tanto los puntos críticos para este caso son $P_1 = (\sqrt{5}, 2 \sqrt{5})$ y $P_2 = (-\sqrt{5}, -2\sqrt{5})$

    \textbf{Caso 2: } $x + 2y = 0$, en este caso tenemos que $x = -2y$, reemplazamos en la 3ra ecuacion y obtenemos que
    \begin{equation*}
        {(-2y)}^{2} + y^2 = 25 \implies 5y^2 = 25 \implies y = \pm \sqrt{5}
    \end{equation*}

    Por lo que los puntos criticos son $P_3 = (-2\sqrt{5}, \sqrt{5})$ y $P_4 = (2\sqrt{5}, -\sqrt{5})$

    Reemplazando en $T$ obtenemos que
    \begin{align*}
        T(\sqrt5, 2\sqrt5) = 4{(\sqrt5)}^2 - 4 \cdot \sqrt{5} \cdot 2\sqrt5 + {(2\sqrt{5})}^2 = 20 - 40 + 20 &= 0\\
        T(-\sqrt5, -2\sqrt5) = 4{(-\sqrt5)}^2 - 4 \cdot (-\sqrt5) \cdot (-2\sqrt5) + {(-2\sqrt5)}^2 = 20 - 40 + 20 &= 0\\
        T(-2\sqrt5, \sqrt5) = 4\cdot {(-2\sqrt5)}^2 -4\cdot(-2\sqrt5)\cdot\sqrt5 + \sqrt5^2 = 80 + 40 + 5 &= 125\\
        T(2\sqrt5, -\sqrt5) = 4 \cdot {(2\sqrt5)}^2 - 4 \cdot (2\sqrt5) \cdot (-\sqrt5) + {(-\sqrt5)}^2 = 80 + 40 + 5 &= 125
    \end{align*}

    Por lo tanto la maxima temperatura que sentira la hormiga sera en de $125$ C en los puntos $(-2\sqrt5, \sqrt5)$ y $(2\sqrt5, -\sqrt5)$. La temperatura minima sera de $0$ C en los puntos $(\sqrt5, 2\sqrt5)$ y $(-\sqrt5, -2\sqrt5)$.
\end{sol}

\begin{prob}
    La temperatura sobre una placa circular $\Omega = \{(x, y) \in \RR^2 \; | \; x^2 + y^2 \leq 1\}$ esta dada por $T(x, y) = 2x^2 + y^2$. Determinar los puntos sobre la placa que estan a mayor y menor temperatura.
\end{prob}

\begin{sol}
    Notemos que la restricción viene dada por que los puntos deben vivir dentro de $\Omega$, para esto se debe cumplir que $x^2 + y^2 \leq 1$. Partamos viendo el caso donde $x^2 + y^2 < 1$. Notemos que la funcion $T$ es de clase $\mathcal{C}^\infty$

    \textbf{Caso 1: } $x^2 + y^2 < 1$, para esto buscaremos los puntos criticos de $T$ mediante el gradiante. Para esto necesitamos que $DT(x, y) = (0,0)$, calculemos la derivada
    \begin{equation*}
        DT(x, y) = \begin{bmatrix}
            \partial_x T & \partial_y T
        \end{bmatrix} = \begin{bmatrix}
            4x & 2y
        \end{bmatrix}
    \end{equation*}

    Luego tenemos que $4x = 0$ e $2y = 0$, por lo tanto el único punto critico es el $(0, 0)$, notemos que $0^2 + 0^2 \leq 1$ y por lo tanto satisface la restricción. Por lo que nuestro primer punto critico es $P_1 = (0, 0)$.

    \textbf{Caso 2: }: $x^2 + y^2 = 1$, usaremos multiplicadores de Lagrange, luego definimos
    \begin{gather*}
        g: \RR^2 \to \RR\\
        (x, y) \mapsto x^2 + y^2 - 1
    \end{gather*}

    Entonces queremos que
    \begin{equation*}
        DT(x, y) = \lambda Dg(x, y)
    \end{equation*}

    Calculemos las derivadas de $g$, pues la derivada de $T$ ya la calculamos.
    \begin{equation*}
        Dg(x, y) = \begin{bmatrix}
            \partial_x g & \partial_y g
        \end{bmatrix} = \begin{bmatrix}
            2x & 2y
        \end{bmatrix}
    \end{equation*}

    Luego nos queda que
    \begin{equation*}
        \begin{bmatrix}
            4x & 2y
        \end{bmatrix} = \begin{bmatrix}
            2\lambda x & 2\lambda y
        \end{bmatrix}
    \end{equation*}

    Es decir tenemos el siguiente sistema de ecuaciones
    \begin{align*}
        4x &= 2\lambda x\\
        2y &= 2 \lambda y\\
        x^2 + y^2 &= 1
    \end{align*}

    De las 2 primeras ecuaciones obtenemos que
    \begin{align*}
        2x(2 - \lambda) &= 0\\
        2y(1 - \lambda) &= 0
    \end{align*}

    \textbf{Caso} $\mathbf{x = 0}$: si $x = 0$ entonces tenemos de la tercera ecuación que $y = \pm 1$, por lo que los puntos críticos asociados a este caso son $P_2 = (0, 1)$, $P_3 = (0, -1)$.

    \textbf{Caso} $\mathbf{\lambda = 2}$: Si $\lambda = 2$, entonces obtenemos de la segunda ecuacion que
    \begin{equation*}
        -2 y = 0 \implies y = 0
    \end{equation*}

    de la tercera ecuación obtenemos que $x = \pm 1$, luego los puntos asociados a este caso son
    \begin{align*}
        P_4 &= (1, 0)\\
        P_5 &= (-1, 0)
    \end{align*}

    \textbf{Caso} $\mathbf{y = 0}$: Este caso es análogo a uno que ya hicimos, pues reemplazando en la tercera ecuación obtenemos que $x = \pm1$, por lo que no obtenemos puntos nuevos.

    \textbf{Caso} $\mathbf{\lambda = 1}$: Reemplazando $\lambda$ en la primera, tenemos que $2x = 0$ y por tanto $x = 0$, luego reemplazando en la tercera obtenemos $y = \pm 1$ por lo que no tenemos nuevos puntos.

    Por lo tanto todos los puntos criticos son 
    \begin{align*}
        P_1 &= (0, 0)\\
        P_2 &= (0, 1)\\
        P_3 &= (0, -1)\\
        P_4 &= (1, 0)\\
        P_5 &= (-1, 0)
    \end{align*}

    Evaluamos todos los puntos en la funcion $T$ para encontrar cual es el minimo y cual es el maximo.

    \begin{align*}
        T(0, 0) &= 0\\
        T(0, 1) &= 1\\
        T(0, -1) &= 1\\
        T(1, 0) &= 2\\
        T(-1, 0) &= 2
    \end{align*}

    Luego la maxima temperatura es $2$ y la minima es $0$.
\end{sol}


\end{document}
