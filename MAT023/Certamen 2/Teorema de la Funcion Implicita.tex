\documentclass[a4paper,oneside,10.5pt]{article}

\usepackage{../../TeXMacros/personal}

\hypersetup{
pdfauthor = {Jorge Bravo},
pdfsubject = {MAT023 2024},
pdftitle = {Teorema de la Funcion Implicita},
pdfkeywords = {},
}

\setlength\parindent{0px}
\begin{document}


\begin{center}
{\Large \textsc{Teorema de la Función Implícita}}\\
\vspace{1em}
\textsc{Ayudante: Jorge Bravo}\\
\end{center}

\section*{Enunciado del Teorema}
\teo[Función Implícita] Sea 
\begin{gather*}
    f : \RR^n \times \RR^m \to \RR^m\\
    (x, y) \mapsto f(x, y)
\end{gather*}
una función de clase $\mathcal{C}^{1}$. Sea $(a, b) \in \RR^n \times \RR^m$ un punto tal que se cumpla que
\begin{equation*}
    f(a, b) = 0
\end{equation*}

y que también se cumpla que $D_{y}f(a, b)$ sea invertible (equivalente a que su determinante sea distinto de $0$). Entonces existen vecindades $\mathcal{U} \subset \RR^n$ del punto $a$ y $\mathcal{V} \subset \RR^m$ del punto $b$ tal que se tiene lo siguiente

\begin{enumerate}
    \item Existe la siguiente funci\'on de clase $\mathcal{C}^1$
    \begin{gather*}
        \varphi : \mathcal{U} \subset \RR^n \to \mathcal{V} \subset \RR^m\\
        x \mapsto y(x)
    \end{gather*}
    tal que se tiene que $f(x, \varphi(x)) = 0, \forall x \in \mathcal{U}$
    \item Se tiene la siguiente igualdad para la derivada de $\varphi$ en el punto $a$
    \begin{equation*}
        D\varphi(a) = -(D_{y}f(a, b))^{-1} Df(a, b)
    \end{equation*}
\end{enumerate}

\section*{Problemas Resueltos}
\prob Muestre que el sistema
\begin{align*}
    x^3 + uy^2 + v &= 0\\
    y^3 + yv + x^2 &= 0
\end{align*}
define a $u$ y $v$ como funciones de $x$ e $y$ en vecindades de los puntos $(x, y) = (0, 1)$ y $(u, v) = (1, -1)$. Sea $w = (1, 1)$, determine la derivada direccional de $v = v(x, y)$ en el punto $(0, 1)$ en la dirección de $w$.

\sol Lo primero que queremos ver es que $u$ y $v$ se definen como funciones implícitas cerca del punto $(x, y, u, v) = (0, 1, 1, -1)$. Notemos que tenemos $2$ ecuaciones, $2$ variables ``conocidas'' (independientes) y $2$ variables ``desconocidas'' (dependientes), por lo tanto nos armamos la siguiente función
\begin{gather*}
    f : \RR^2 \times \RR^2 \to \RR^2\\
    ((x, y), (u, v)) \mapsto (x^3 + uy^2 + v, y^3 + yv + x^2)
\end{gather*}

Darse cuenta que $f$ es de clase $\mathcal{C}^{-1}$ por álgebra de funciones $\mathcal{C}^1$. Notemos además que $f((0, 1), (1, -1)) = (0, 0)$ y veamos cuales derivadas tenemos que calcular para aplicar el teorema de la función implícita.

\begin{equation*}
    Df((0,1), (1, -1))_{2 \times 4} = \left[\begin{array}{cc|cc}
        \partial_x f_1 & \partial_y f_1 & \partial_u f_1 & \partial_v f_1\\
        \partial_x f_2 & \partial_y f_2 & \partial_u f_2 & \partial_v f_2\\
    \end{array}
    \right]_{((x, y), (u, v)) = ((0, 1), (1 -1))}
\end{equation*}

Como despues nos piden obtener la derivada direccional de $v(x, y)$ necesitaremos todas las derivadas asi que calculemos todas de al tiro.

\begin{equation}
    Df((0,1), (1, -1))_{2 \times 4} = \left[\begin{array}{cc|cc}
        3x^2 & 2uy & y^2 & 1\\
        2x & 3y^2 + v & 0 & y
    \end{array}
    \right]_{((x, y), (u, v)) = ((0, 1), (1 -1))} = \left[\begin{array}{cc|cc}
        0 & 2 & 1 & 1\\
        0 & 2 & 0 & 1
    \end{array}
    \right]
\end{equation}

Recordemos que siempre, si dejamos las variables que no conocemos a la derecha, la matriz de la derecha sera $D_{(u, v)}f((0, 1), (1, -1))$, cuando le calculamos el determinante a esa matriz, es decir a $\begin{bmatrix}1 & 1\\ 0 & 1\end{bmatrix}$, nos da $1$ y por tanto es invertible.

Por lo tanto, por el teorema de la función implícita, existen vecindades $\mathcal{U} \subset \RR^2$ de $(0, 1)$ y $\mathcal{V} \subset \RR^2$ de $(1, -1)$ y la siguiente función
\begin{gather*}
    \varphi : \mathcal{U} \subset \RR^2 \to \mathcal{V} \subset \RR^2\\
    (x, y) \mapsto (u(x, y), v(x, y))
\end{gather*}
La cual cumple que
\begin{equation*}
f((x, y), (\varphi(x, y)) = 0, \forall (x, y) \in \mathcal{U}
\end{equation*}
Es decir que $\varphi$ define de manera implícita a $u$ y $v$ en función de $x$ e $y$.

Tambien sabemos que la derivada de $\varphi$ en el punto $(0, 1)$ viene dada por
\begin{equation}
\begin{bmatrix}
    \partial_x u & \partial_y u\\
    \partial_x v & \partial_y v
\end{bmatrix}_{(x, y) = (0, 1)}= D\varphi(0, 1) = -(D_{(u, v)}f((0, 1), (1, -1)))^{-1} D_{(x, y)}f((0, 1), (1, -1))
\end{equation}

Ahora queremos calcular la derivada direccional de $v(x, y)$ en el punto $(0, 1)$ en la dirección de $w$. Para esto recordemos que la derivada direccional de una función diferenciables viene dada por

\begin{equation}
    \frac{\partial v}{\partial w}(0, 1) = \nabla v(0, 1) \cdot \frac{w}{||w||} = \begin{bmatrix} \partial_x v(0, 1) & \partial_y v(0, 1) \end{bmatrix} \cdot \frac{\begin{bmatrix}
        1 & 1
    \end{bmatrix}}{\sqrt{1^2 + 1^2}} = \frac{1}{\sqrt2}(\partial_x v(0, 1) + \partial_y v(0, 1))
\end{equation}

Por lo tanto solo nos falta conocer $\partial_x v(0, 1)$ y $\partial_y v(0, 1)$, pero estos los podemos obtener desde la derivada de $\varphi$ desde la ecuación $(2)$. Pero esas matrices ya las calculamos antes en la ecuación $(1)$, por lo que solo tenemos que multiplicar y ver que nos da

\begin{equation*}
    D\varphi(0, 1) = -(D_{(u, v)}f((0, 1), (1, -1)))^{-1} D_{(x, y)}f((0, 1), (1, -1)) = - \begin{bmatrix}
        1 & -1\\ 0 & 1
    \end{bmatrix}\begin{bmatrix}
        0 & 2\\ 0 & 2
    \end{bmatrix} = \begin{bmatrix}
        0 & 0\\ 0 & 2
    \end{bmatrix}
\end{equation*}

Por lo tanto recordamos de $(2)$ que
\begin{equation*}
    \begin{bmatrix}
    \partial_x u & \partial_y u\\
    \partial_x v & \partial_y v
\end{bmatrix}_{(x, y) = (0, 1)}= D\varphi(0, 1) =\begin{bmatrix}
        0 & 4\\ 0 & 2
    \end{bmatrix}
\end{equation*}

Por lo tanto $\partial_x v(0, 1) = 0$ y $\partial_y v(0,1) = 2$

Reemplazando en $(3)$ entonces obtenemos que
\begin{equation*}
    \frac{\partial v}{\partial w}(0, 1) = \frac{1}{\sqrt2}(0 + 2) = \frac{2}{\sqrt{2}} = \frac{2}{\sqrt2} \cdot \frac{\sqrt2}{\sqrt2} = \sqrt{2}
\end{equation*}

\prob Considere el siguiente sistema de ecuaciones
\begin{align*}
    xu^2 + 2y^2v^2 + uvz &= 1\\
    vx^2 - y^2 + uv^2 &= 0
\end{align*}

Muestre que el sistema define funciones implícitas $u = u(x, y, z)$ y $v = (x,y,z)$, en una vecindad del punto $(0, -1, 1, 1, -1)$. Justifique

\sol Sabemos que cuando queremos ver que un sistema define de manera implícita a ciertas variables en funciones de otras, lo mejor que podemos hacer es usar el teorema de la función implícita. Notemos que ``conocemos'' $3$ variables $(x, y, z)$ y ``desconocemos'' a $2$, $(u, v)$. Por lo tanto definimos la siguiente función

\begin{gather*}
    f: \RR^3 \times \RR^2 \to \RR^2\\
    ((x, y, z), (u, v)) \mapsto (xu^2 + 2y^2v^2 + uvz - 1, vx^2 - y^2 + uv^2)
\end{gather*}
Notemos que esta función es de clase $\mathcal{C}^1$ pues es un polinomio en $5$ variables (otra opción es argumentar por álgebra de funciones $\mathcal{C}^1$).

Luego evaluamos en el punto para ver que se satisface que $f((0, -1, 1), (1, -1)) = (0, 0)$
\begin{equation*}
    f((0, -1, 1), (1, -1)) = (0 + 2 - 1 - 1, -1 + 1) = (0, 0)
\end{equation*}

Por lo tanto nos falta verificar que la derivada con respecto a las variables que no conocemos tiene determinante distinto de $0$, recordemos quien es esta derivada en el punto $((0, -1, 1), (1, -1))$
\begin{equation*}
    Df((0, -1, 1), (1, -1)_{2 \times 5} = \left[\begin{array}{ccc|cc}
        \partial_x f_1 & \partial_y f_1 & \partial_z f_1 & \partial_u f_1 & \partial_v f_1\\
        \partial_x f_2 & \partial_y f_2 & \partial_z f_2 & \partial_u f_2 & \partial_v f_2\\
    \end{array}
    \right]_{((x, y, z), (u, v)) = ((0, -1, 1), (1 -1))}
\end{equation*}

Recordemos que la derivada que nos interesa es la matriz que queda a la derecha (siempre que seamos ordenados y definamos a $f$ de la manera correcta), la cual llamamos $D_{(u, v)}f$, dado que no necesitaremos las derivadas de $u(x, y, z)$ ni $v(x, y, z)$ no es necesario calcular las derivadas de la izquierda, por lo que solo calcularemos la que nos interesa.

\begin{align*}
    D_{(u, v)}f((0, -1, 1), (1 -1)) &= \begin{bmatrix}
        \partial_u f_1 & \partial_v f_1\\
        \partial_u f_2 & \partial_v f_2
    \end{bmatrix}_{((x, y, z), (u, v)) = ((0, -1, 1), (1 -1))}\\
    &= \begin{bmatrix}
            2xu + vz & 4y^2v + uz\\
            v^2 & x^2 + 2uv
        \end{bmatrix}_{((x, y, z), (u, v)) = ((0, -1, 1), (1 -1))}\\
    &= \begin{bmatrix}
        -1 & -3\\
        1 & -2
        \end{bmatrix}
\end{align*}

Luego queremos saber si esta matriz es invertible, para poder aplicar el teorema de la función implícita, por lo que calcularemos su determinante para ver que es distinto de $0$.

\begin{equation*}
    \det \begin{bmatrix}
        -1 & -3\\
        1 & -2
        \end{bmatrix} = 2 + 3 = 5
\end{equation*}

Por lo tanto es invertible. Por el teorema de la Función Implícita existen vecindades $\mathcal{U} \subset \RR^3$ de $(0, -1, 1)$ y $\mathcal{V} \subset \RR^2$ de $(1, -1)$ de tal forma que existe la función
\begin{gather*}
    \varphi : \mathcal{U} \subset \RR^3 \to \mathcal{V} \subset \RR^2\\
    (x, y, z) \mapsto (u(x, y, z), v(x, y, z))
\end{gather*}

que cumple que $f((x, y, z), \varphi(x, y, z)) = 0, \forall (x, y, z) \in \mathcal{U}$, en otras palabras, $f$ define a $u$ y $v$ de manera implícita en función de $x, y, z$.
\end{document}
